% +++
% latex = "lualatex"
% bibtex = "pbibtex"
% 
% sequence = [ "latex", "bibtex",  "latex", "latex", "dvipdf" ]
% max_repeat = 7
% +++
%#!llmk main.tex
\documentclass[lualatex, paper=a4, number_of_lines=45]{jlreq}
\usepackage{mysettings}
\usepackage{docmute}
\title{\qedsymbol を証明終了マークとして用いてみよう.}
\author{そくらてす}
\date{\today}
\begin{document}
\maketitle

この文書では\qedsymbol を証明終了マークとして用いる方法について書く\footnote{\qedsymbol は``wwwww(which was what we wanted)''を表している.}.

\section{設定方法}
前提として,amsthm パッケージを用いて文書作成をしている人向けに書いてある.

\qedsymbol を証明終了マークとして用いる手順は次のとおりである.
\begin{enumerate}
 \item bxcoloremoji パッケージを使えるようにする.
       \begin{enumerate}
	\item このパッケージの設定方法は\url{https://github.com/zr-tex8r/BXcoloremoji}を参照のこと.
	\item 雑に説明すると \verb|$TEXMF/tex/latex/BXcoloremoji| に git clone すればよい.
       \end{enumerate}
 \item プリアンブルに次を追加する.
\begin{verbatim}
\usepackage{bxcoloremoji} 
\renewcommand{\qedsymbol}{\coloremoji*{<草の絵文字>}}
\end{verbatim}
\verb|<草の絵文字> |には \coloremoji*{🌿}を入れる.
ここを\coloremoji*{🪴}に変えれば\coloremoji*{🪴}が証明終了のマークになる.
\end{enumerate}

\section{例}
前の節の設定をしておくと次のように証明終了マークが\qedsymbol になる.
\begin{lemma}[K\"{o}nig の補題]
 有限の枝分かれしか持たない無限木には無限道が存在する.
\end{lemma}
\begin{proof}
  $\mathtree{D}$ を有限の枝分かれしか持たない無限木とする.

 次の条件\cref{item:Koenig-child}と\cref{item:Koenig-set}を満たす
 $\mathtree{D}$の頂点$\sigma_{i}$と$\mathtree{D}$ の頂点の無限集合 $P_{i}$の組からなる
 無限列 $\mathsequence{\mleft(\sigma_{i}, P_{i} \mright)}{i\in\mathnat}$ が
 存在することを示す.
  \begin{enumerate}
   \item $\sigma_{i}$は$\sigma_{i-1}$の子(ただし,$1\leq i$).
	 \label{item:Koenig-child}
   \item $P_{i}$の元は全て$\sigma_{i}$の子孫.
	 \label{item:Koenig-set}
  \end{enumerate}

  $\mathsequence{\mleft(\sigma_{i}, P_{i} \mright)}{i\in\mathnat}$ を
 帰納的に構成することにより示す.

  $i=0$のとき.$\sigma_{0}$ を$\mathtree{D}$の根,$P_{0}=\mathtree{D}$と定義する.
  これらが\cref{item:Koenig-child} と\cref{item:Koenig-set}を満たすのは明らかである.

  $i>0$のとき.
  $P_{i-1}$は無限集合であり,また$\sigma_{i-1}$の子は有限であるから,
  $\sigma_{i-1}$ の子$\sigma$でその子孫の集合が無限であるものが少なくとも1つ存在する.
  $\sigma_{i}$ を $\sigma$,$P_{i}$を$\sigma_{i}$の子孫全体とする.
 これらが\cref{item:Koenig-child} と\cref{item:Koenig-set}を満たすのは明らかである.

 以上により$\mathsequence{\mleft(\sigma_{i}, P_{i} \mright)}{i\in\mathnat}$ 
 は構成できた.
 このとき, $\mathsequence{\sigma_{i}}{i\in\mathnat}$は無限道である.
 よって,$\mathtree{D}$には無限道が存在する.
\end{proof}

\end{document}